\documentclass[12pt]{article}

\usepackage{fullpage}
\usepackage{multicol,multirow}
\usepackage{tabularx}
\usepackage{ulem}
\usepackage[utf8]{inputenc}
\usepackage[russian]{babel}
\usepackage{amsmath}
\usepackage{amssymb}

\usepackage{titlesec}

\titleformat{\section}
  {\normalfont\Large\bfseries}{\thesection.}{0.3em}{}

\titleformat{\subsection}
  {\normalfont\large\bfseries}{\thesubsection.}{0.3em}{}

\titlespacing{\section}{0pt}{*2}{*2}
\titlespacing{\subsection}{0pt}{*1}{*1}
\titlespacing{\subsubsection}{0pt}{*0}{*0}
\usepackage{listings}
\lstloadlanguages{Lisp}
\lstset{extendedchars=false,
	breaklines=true,
	breakatwhitespace=true,
	keepspaces = true,
	tabsize=2
}
\begin{document}


\section*{Отчет по лабораторной работе №\,1 
по курсу \guillemotleft  Функциональное программирование\guillemotright}
\begin{flushright}
Студент группы 8О-307Б-18 МАИ \textit{Токарев Никита}, \textnumero 21 по списку \\
\makebox[7cm]{Контакты: {\tt tokarevnikita08@mail.ru} \hfill} \\
\makebox[7cm]{Работа выполнена: 15.03.2021 \hfill} \\
\ \\
Преподаватель: Иванов Дмитрий Анатольевич, доц. каф. 806 \\
\makebox[7cm]{Отчет сдан: \hfill} \\
\makebox[7cm]{Итоговая оценка: \hfill} \\
\makebox[7cm]{Подпись преподавателя: \hfill} \\

\end{flushright}

\section{Тема работы}
Примитивные функции и особые операторы в Common Lisp.

\section{Цель работы}
Изучить примитивные функции и особые операторы в Common Lisp.

\section{Задание (вариант № 1.19 )}
Запрограммируйте на языке Коммон Лисп функцию-предикат с тремя параметрами - действительными положительными числами a, b, c. Функция должна возвращать T (истину), если существует прямоугольный треугольник с длинами сторон a, b и c.

\section{Оборудование студента}
Процессор Intel® Core™ i3-5005U CPU @ 2.00GHz × 4, память: 3,8 Gb, разрядность системы: 64.

\section{Программное обеспечение}
UBUNTU 18.04.5 LTS, компилятор sbcl

\section{Идея, метод, алгоритм}
Идея в том, чтобы посчитать по теореме пифагора предпологаемую длину гипотенузы и сравнить с максимальной по длине стороной. В данной программе реализованы две ключевые функции (AAL a b c) и (right-angled a b с).
\begin{itemize}
\setlength{\itemsep}{-1mm} % уменьшает расстояние между элементами списка
\item (AAL a b c) - поиск максимальной по длине стороны треугольника и возврат суммы квадратов оставшихся двух сторон
\item (right-angled a b с) - выборка из всех возможных успешных вариантов.
\end{itemize}

\section{Сценарий выполнения работы}
\begin{itemize}
\setlength{\itemsep}{-1mm}
\item Анализ возможных реализаций поставленной задачи на common Lisp
\item Изучение синтаксиса и основных функций common Lisp
\item Реализация поставленной задачи на common Lisp
\end{itemize}
\section{Распечатка программы и её результаты}

\subsection{Исходный код}
\begin{verbatim}
(defun square (x) (* x x))

(defun right-angled (a b c)
    (cond
        ((equalp (square a) (AAL a b c)) T)
        ((equalp (square b) (AAL a b c)) T)
        ((equalp (square c) (AAL a b c)) T)
    )
)

(defun AAL (a b c)
    (if (> a b)
        (if(> a c)
            (+ (square b) (square c))
            (+ (square a) (square b))
        )
        (if(> b c)
            (+ (square a) (square c))
            (+ (square a) (square b))
        )
    )
)
\end{verbatim}
%\lstinputlisting{./lab2.lisp}

\subsection{Результаты работы}
\begin{verbatim}
* (right-angled 5.0 3.0 4.0)

T
* (right-angled  2.2 3.3 4.2)

NIL
* (right-angled 3.84187454246 5.0 3.2)

T
* (right-angled 3.0 3.0 3.0)

NIL
\end{verbatim}
%\lstinputlisting{./log2.lisp}

\section{Дневник отладки}
\begin{tabular}{|p{50pt}|p{130pt}|p{130pt}|p{70pt}|}
\hline
Дата & Событие & Действие по исправлению & Примечание \\ \hline
15.03.2021 & Функция AAL возвращала ошибку:caught ERROR:illegal function call & Добавил функцию right-angled, где происходила выборка. AAL сделал вспомогательной.  &\\
\hline
\end{tabular}

\section{Замечания автора по существу работы}
Замечаний нет.

\section{Выводы}
Данная работа позволила отойти от стандартного императивного программирования и взглянуть на решение поставленной задачи  в функциональной парадигме. Данный алгоритм тривиален и работает за константное время. Также хотелось бы отметить, что в ходе данной работы познакомился с синтаксисом common Lisp и некоторыми основными функциями common Lisp.

\end{document}
