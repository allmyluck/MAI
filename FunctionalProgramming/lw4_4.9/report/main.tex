\documentclass[12pt]{article}

\usepackage{fullpage}
\usepackage{multicol,multirow}
\usepackage{tabularx}
\usepackage{ulem}
\usepackage[utf8]{inputenc}
\usepackage[russian]{babel}
\usepackage{amsmath}
\usepackage{amssymb}

\usepackage{titlesec}

\titleformat{\section}
  {\normalfont\Large\bfseries}{\thesection.}{0.3em}{}

\titleformat{\subsection}
  {\normalfont\large\bfseries}{\thesubsection.}{0.3em}{}

\titlespacing{\section}{0pt}{*2}{*2}
\titlespacing{\subsection}{0pt}{*1}{*1}
\titlespacing{\subsubsection}{0pt}{*0}{*0}
\usepackage{listings}
\lstloadlanguages{Lisp}
\lstset{extendedchars=false,
	breaklines=true,
	breakatwhitespace=true,
	keepspaces = true,
	tabsize=2
}
\begin{document}


\section*{Отчет по лабораторной работе №\,4 
по курсу \guillemotleft  Функциональное программирование\guillemotright}
\begin{flushright}
Студент группы 8О-307Б-18 МАИ \textit{Токарев Никита}, \textnumero 21 по списку \\
\makebox[7cm]{Контакты: {\tt tokarevnikita08@mail.ru} \hfill} \\
\makebox[7cm]{Работа выполнена: 11.04.2021 \hfill} \\
\ \\
Преподаватель: Иванов Дмитрий Анатольевич, доц. каф. 806 \\
\makebox[7cm]{Отчет сдан: \hfill} \\
\makebox[7cm]{Итоговая оценка: \hfill} \\
\makebox[7cm]{Подпись преподавателя: \hfill} \\

\end{flushright}

\section{Тема работы}
Знаки и строки.

\section{Цель работы}
Изучить  знаки и строки, а также методы работы с ними в Коммон Лисп.

\section{Задание (вариант № 4.9)}
Запрограммировать на языке Коммон Лисп функцию с двумя параметрами: char - знак, sentence - строка предложения. Функция должна подсчитать число вхождений знака char в последнее слово предложения sentence. Сравнение как латинских букв, так и русских должно быть регистро-независимым.


\section{Оборудование студента}
Процессор Intel® Core™ i3-5005U CPU @ 2.00GHz × 4, память: 3,8 Gb, разрядность системы: 64.

\section{Программное обеспечение}
UBUNTU 18.04.5 LTS, компилятор sbcl

\section{Идея, метод, алгоритм}
Идея в том, чтобы начать поиск от последнего элемента игнорируя различные знаки(.,?,",!) в конце предложения и до первого пробела или конца предложения(если предложение состоит из одного слова). В программе одна основная функции:
\begin{itemize}
\setlength{\itemsep}{-1mm} % уменьшает расстояние между элементами списка
\item (last-word-char-count ch str) - В данной функции происходит поэлементное сравнение элементов с условиями выхода:следующий элемент пробел или начало предложения(если предложение состоит из одного слова).
\end{itemize}

\section{Сценарий выполнения работы}
\begin{itemize}
\setlength{\itemsep}{-1mm}
\item Анализ возможных реализаций поставленной задачи на common Lisp
\item Изучение синтаксиса и основных функций работы со списками common Lisp
\item Реализация поставленной задачи на common Lisp
\end{itemize}
\section{Распечатка программы и её результаты}

\subsection{Исходный код}
Значения тестовых функций представлено в исходном коде.
\begin{verbatim}
(defun whitespace-char(ch)
    (member ch '(#\Space #\Tab #\Newline))
)

(defun endSentence-char(ch)
    (member ch '(#\! #\? #\. #\"))
)

(defun last-word-char-count(ch str)
    (let
        (
            (res 0)
            (cur-ch nil)
        )
        (loop for i from (- (length str) 1) downto 0 do
            (setq cur-ch (char str i))	
            (if (and (char-equal ch cur-ch) (not (endSentence-char cur-ch)))
                (setq res (+ res 1))
            )	
            (if (or (whitespace-char cur-ch) (equalp i 0))
                (return res)
            )
        )
    )
)

;(last-word-char-count #\Ы "К долинам,  покоем объятым,  ему не сойти с высоты.")
;(last-word-char-count #\А "\"Почему здесь никого нет, - удивилась Аленка,
- я пришла раньше или опоздала?\"")
;(last-word-char-count #\а "\"Почему здесь никого нет, - удивилась Аленка,
- я пришла раньше или опоздала?\"")
;(last-word-char-count #\о "Много.")
;(last-word-char-count #\к "Креветка.")
;(last-word-char-count #\к "Креветка!?")
;(last-word-char-count #\к "Двойные кавычки активно используются в 
русском языке в машинном тексте!?")
;(last-word-char-count #\у "Двойные кавычки активно используются в
русском языке в машинном тексте!?")
;(last-word-char-count #\f "Hello, my friend!")

\end{verbatim}
%\lstinputlisting{./lab2.lisp}

\subsection{Результаты работы}
\begin{verbatim}
* (last-word-char-count #\о "Много.")

2
* (last-word-char-count #\к "Креветка!?")

2
* (last-word-char-count #\Ы "К долинам,  покоем объятым,  ему не сойти с высоты.")

2
* (last-word-char-count #\у "Двойные кавычки активно используются
в русском языке в машинном тексте!?")

0
* (last-word-char-count #\а "\"Почему здесь никого нет, - удивилась Аленка,
- я пришла раньше или опоздала?\"")

2
* (last-word-char-count #\А "\"Почему здесь никого нет, - удивилась Аленка,
- я пришла раньше или опоздала?\"")

2
* (last-word-char-count #\f "Hello, my friend!")

1
\end{verbatim}
%\lstinputlisting{./log2.lisp}

\section{Дневник отладки}
\begin{tabular}{|p{50pt}|p{130pt}|p{130pt}|p{70pt}|}
\hline
Дата & Событие & Действие по исправлению & Примечание \\ \hline
\end{tabular}

\section{Замечания автора по существу работы}
Замечаний нет.

\section{Выводы}
В ходе данной работы мне удалось познакомиться со встроенными функциями/интсрументами для работы со знаками и строками. Со строками я был знаком и ранее, однако было довольно интересно увидеть применение такой структуры данных в common Lisp. В моей программе алгоритм работает за линейное время O(n), где n - длина исследуемого слова.

\end{document}
