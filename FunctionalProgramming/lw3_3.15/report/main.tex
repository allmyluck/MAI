\documentclass[12pt]{article}

\usepackage{fullpage}
\usepackage{multicol,multirow}
\usepackage{tabularx}
\usepackage{ulem}
\usepackage[utf8]{inputenc}
\usepackage[russian]{babel}
\usepackage{amsmath}
\usepackage{amssymb}

\usepackage{titlesec}

\titleformat{\section}
  {\normalfont\Large\bfseries}{\thesection.}{0.3em}{}

\titleformat{\subsection}
  {\normalfont\large\bfseries}{\thesubsection.}{0.3em}{}

\titlespacing{\section}{0pt}{*2}{*2}
\titlespacing{\subsection}{0pt}{*1}{*1}
\titlespacing{\subsubsection}{0pt}{*0}{*0}
\usepackage{listings}
\lstloadlanguages{Lisp}
\lstset{extendedchars=false,
	breaklines=true,
	breakatwhitespace=true,
	keepspaces = true,
	tabsize=2
}
\begin{document}


\section*{Отчет по лабораторной работе №\,3 
по курсу \guillemotleft  Функциональное программирование\guillemotright}
\begin{flushright}
Студент группы 8О-307Б-18 МАИ \textit{Токарев Никита}, \textnumero 21 по списку \\
\makebox[7cm]{Контакты: {\tt tokarevnikita08@mail.ru} \hfill} \\
\makebox[7cm]{Работа выполнена: 06.04.2021 \hfill} \\
\ \\
Преподаватель: Иванов Дмитрий Анатольевич, доц. каф. 806 \\
\makebox[7cm]{Отчет сдан: \hfill} \\
\makebox[7cm]{Итоговая оценка: \hfill} \\
\makebox[7cm]{Подпись преподавателя: \hfill} \\

\end{flushright}

\section{Тема работы}
Последовательности, массивы и управляющие конструкции Коммон Лисп.

\section{Цель работы}
Изучить  Последовательности, массивы и управляющие конструкции Коммон Лисп.

\section{Задание (вариант № 3.15)}
Запрограммировать на языке Коммон Лисп функцию, принимающую в качестве аргументов следующие:\newline
    A - двумерный массив, представляющий действительную матрицу размера m*n,\newline
    u - вектор (строка) действительных чисел длины n,\newline
    v - вектор (столбец) действительных чисел длины m+1,\newline
    i - номер строки, 1<=i<=m,\newline
    j - номер столбца, 1<=j<=n.\newline

Функция должна возвращать новую матрицу размера (m+1)*(n+1), полученную из A путём вставки элементов u в качестве новой строки после строки с номером i. последующей вставки элементов v в качестве нового столбца после столбца с номером j.


\section{Оборудование студента}
Процессор Intel® Core™ i3-5005U CPU @ 2.00GHz × 4, память: 3,8 Gb, разрядность системы: 64.

\section{Программное обеспечение}
UBUNTU 18.04.5 LTS, компилятор sbcl

\section{Идея, метод, алгоритм}
Метод состоит из двух итераций. Идея в том, чтобы для начала вставить u-вектор на место i+1 строки матрицы b. Затем же в полученную матрицу на место j+1 столбца вставить v-вектор. Мне предстоит разбить матрицу на составные части и по новому собрать ее. В программе две основные функции:
\begin{itemize}
\setlength{\itemsep}{-1mm} % уменьшает расстояние между элементами списка
\item (line-extend-matrix a k v) - В данной функции происходит создание матрицы b и копирование элементов матрицы до элемента a[k][...]. Затем же вставка вектора строки на позицию b[k][...] и копирование остальных элементов, если оно необходимо. 
\item (sum-product1 list1 list2) - рекурсивная функция: произведение n-х элементов списков  list1 и list2  суммируется с результатом данной вызванной функции, где входными данными являются списки: list1 = list1(n + 1,...), list2 = list2(n + 1,...).
\end{itemize}

\section{Сценарий выполнения работы}
\begin{itemize}
\setlength{\itemsep}{-1mm}
\item Анализ возможных реализаций поставленной задачи на common Lisp
\item Изучение синтаксиса и основных функций работы со списками common Lisp
\item Реализация поставленной задачи на common Lisp
\end{itemize}
\section{Распечатка программы и её результаты}

\subsection{Исходный код}
Значения тестовых глобальных переменных представлено в исходном коде.
\begin{verbatim}
(defun column-extend-matrix (a k u)
  (let ((m (array-dimension a 0))
        (n (array-dimension a 1)))
    (let ((b (make-array (list m (+ 1 n))))) ; b = Matrix(* m (1 + n)) 
      (dotimes (i m)
        (dotimes (j k)
          (setf (aref b i j) (aref a i j)))) ; b[i][j] = a[i][j]
      (loop
        :for i :below m
        :do (setf (aref b i k) (aref u i))) ; b[i][k] = u[i]
      (dotimes (i m)
        (loop :for j :from k :to (- n 1) :do 
          (setf (aref b i (+ 1 j)) (aref a i j)))) ; b[i][1 + j] = a[i][j]
      b)))

(defun line-extend-matrix (a k v)
  (let ((m (array-dimension a 0))
        (n (array-dimension a 1)))
    (let ((b (make-array (list (+ 1 m) n)))) ; b = Matrix(* (1 + m) n)
      (dotimes (i k)
        (dotimes (j n)
          (setf (aref b i j) (aref a i j)))) ; b[i][j] = a[i][j]
      (loop
        :for j :below n
        :do (setf (aref b k j) (aref v j))) ; b[k][j] = v[j], j : 0 to n
      (dotimes (j n)
        (loop :for i :from k :to (- m 1) :do
          (setf (aref b (+ 1 i) j) (aref a i j)))) ; b[i + 1][j] = a[i][j]
      b)))

(defun extend-matrix (a u v i j)
  (array-dimension a 0)
  (array-dimension a 1)
  (if (and (and (<= i (array-dimension a 0)) (<= j (array-dimension a 1))) (and (> i 0) (> j 0))) (column-extend-matrix (line-extend-matrix a i u) j v))
)
;test_variables
(defvar m1 (make-array '(3 3) :initial-contents '((0 0 0 ) (0 0 0) (0 0 0))))
(defvar u1 #(1 1 1))
(defvar v1 #(2 2 2 2))
(defvar m2 (make-array '(2 4) :initial-contents '((0 0 0 0) (0 0 0 0))))
(defvar u2 #(1 1 1 1))
(defvar v2 #(2 2 2))
(defvar m3 (make-array '(3 4) :initial-contents '((0 0 0 0) (0 0 0 0) (0 0 0 0))))
(defvar u3 #(1 1 1 1))
(defvar v3 #(2 2 2 2))
(defvar m4 (make-array '(1 1) :initial-contents '((0))))
(defvar u4 #(1))
(defvar v4 #(2 2))
;(extend-matrix m1 u1 v1 0 0)
;(extend-matrix m2 u2 v2 1 1)
;(extend-matrix m3 u3 v3 2 3)
;(extend-matrix m4 u4 v4 0 0)
;(extend-matrix m4 u4 v4 1 1)
;(extend-matrix m4 u4 v4 2 2)
\end{verbatim}
%\lstinputlisting{./lab2.lisp}

\subsection{Результаты работы}
\begin{verbatim}
* (extend-matrix m1 u1 v1 0 0)

NIL
* (extend-matrix m1 u1 v1 1 1)

#2A((0 2 0 0) (1 2 1 1) (0 2 0 0) (0 2 0 0))
* (extend-matrix m3 u3 v3 2 3)

#2A((0 0 0 2 0) (0 0 0 2 0) (1 1 1 2 1) (0 0 0 2 0))
* (extend-matrix m4 u4 v4 2 2)

NIL
* (extend-matrix m4 u4 v4 1 1)

#2A((0 2) (1 2))
\end{verbatim}
%\lstinputlisting{./log2.lisp}

\section{Дневник отладки}
\begin{tabular}{|p{50pt}|p{130pt}|p{130pt}|p{70pt}|}
\hline
Дата & Событие & Действие по исправлению & Примечание \\ \hline
04.04.2021 & Вставка u и v на позицию i и j соответсвенно. По условию вставка происходит после i и j.  & Изменил входные данный у функции: (column-extend-matrix (line-extend-matrix a (+ 1 i) u) (+ 1 j) v) . &\\
\hline
06.04.2021 & i >=0 и j>=0  & Добавил ограничение, такое что 0<i<=m,0<j<=n в (defun extend-matrix a u v i j) &\\
\hline
\end{tabular}

\section{Замечания автора по существу работы}
Замечаний нет.

\section{Выводы}
В ходе данной работы мне удалось познакомиться со встроенными функциями/интсрументами, а также управляющими конструкциями коммон лисп с помощью которых мне удалось реализовать классический обход по матрице, используя циклы. 

\end{document}
