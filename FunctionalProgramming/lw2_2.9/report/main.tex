\documentclass[12pt]{article}

\usepackage{fullpage}
\usepackage{multicol,multirow}
\usepackage{tabularx}
\usepackage{ulem}
\usepackage[utf8]{inputenc}
\usepackage[russian]{babel}
\usepackage{amsmath}
\usepackage{amssymb}

\usepackage{titlesec}

\titleformat{\section}
  {\normalfont\Large\bfseries}{\thesection.}{0.3em}{}

\titleformat{\subsection}
  {\normalfont\large\bfseries}{\thesubsection.}{0.3em}{}

\titlespacing{\section}{0pt}{*2}{*2}
\titlespacing{\subsection}{0pt}{*1}{*1}
\titlespacing{\subsubsection}{0pt}{*0}{*0}
\usepackage{listings}
\lstloadlanguages{Lisp}
\lstset{extendedchars=false,
	breaklines=true,
	breakatwhitespace=true,
	keepspaces = true,
	tabsize=2
}
\begin{document}


\section*{Отчет по лабораторной работе №\,2 
по курсу \guillemotleft  Функциональное программирование\guillemotright}
\begin{flushright}
Студент группы 8О-307Б-18 МАИ \textit{Токарев Никита}, \textnumero 21 по списку \\
\makebox[7cm]{Контакты: {\tt tokarevnikita08@mail.ru} \hfill} \\
\makebox[7cm]{Работа выполнена: 21.03.2021 \hfill} \\
\ \\
Преподаватель: Иванов Дмитрий Анатольевич, доц. каф. 806 \\
\makebox[7cm]{Отчет сдан: \hfill} \\
\makebox[7cm]{Итоговая оценка: \hfill} \\
\makebox[7cm]{Подпись преподавателя: \hfill} \\

\end{flushright}

\section{Тема работы}
Простейшие функции работы со списками common Lisp.

\section{Цель работы}
Изучить  простейшие функции работы со списками common Lisp.

\section{Задание (вариант № 2.9)}
Дан список действительных чисел ($x_{1}$ ... $x_{n}$), \(n \geqslant 2\).
Запрограммируйте рекурсивно на языке common Lisp функцию, вычисляющую выражение вида:\\

$(x_{1} * x_{n}) + (x_{2} * x_{n-1}) + ... + (x_{n} * x_{1})$.

\section{Оборудование студента}
Процессор Intel® Core™ i3-5005U CPU @ 2.00GHz × 4, память: 3,8 Gb, разрядность системы: 64.

\section{Программное обеспечение}
UBUNTU 18.04.5 LTS, компилятор sbcl

\section{Идея, метод, алгоритм}
Идея в том, чтобы посчитать произведение n-го элемента списка list и реверсированного списка list и просуммировать полученные значения, используя рекурсию. В данной программе реализованы две ключевые функции (sum-product1 list1 list2) и (sum-product2 list).
\begin{itemize}
\setlength{\itemsep}{-1mm} % уменьшает расстояние между элементами списка
\item (sum-product2 list) - вызов функции (sum-product1 list1 list2), где list = list1, list2 = (reverse list), при это длина входного списка не менее 2 и элементы списка действительные числа.
\item (sum-product1 list1 list2) - рекурсивная функция: произведение n-х элементов списков  list1 и list2  суммируется с результатом данной вызванной функции, где входными данными являются списки: list1 = list1(n + 1,...), list2 = list2(n + 1,...).
\end{itemize}

\section{Сценарий выполнения работы}
\begin{itemize}
\setlength{\itemsep}{-1mm}
\item Анализ возможных реализаций поставленной задачи на common Lisp
\item Изучение синтаксиса и основных функций работы со списками common Lisp
\item Реализация поставленной задачи на common Lisp
\end{itemize}
\section{Распечатка программы и её результаты}

\subsection{Исходный код}
\begin{verbatim}
(defun sum-product1 (list1 list2)
    (cond 
        ((null list1) 0)
        (t (+ (* (first list1)(first list2))
            (sum-product1 (rest list1)(rest list2))))
    )
)

(defun sum-product2 (list)
    (if (>= (list-length list) 2)
        (sum-product1 list (reverse list))
    )
)
\end{verbatim}
%\lstinputlisting{./lab2.lisp}

\subsection{Результаты работы}
\begin{verbatim}
* (sum-product2 '(1 2))

4
* (sum-product2 '(1 2 3 4 5))

35
* (sum-product2 '(1 -2 1 -5 7))

35
* (sum-product2 '(1))

NIL
* (sum-product2 '(1 2 1 2 1 2 1 2 1 2))

20
* (sum-product2 '())

NIL
\end{verbatim}
%\lstinputlisting{./log2.lisp}

\section{Дневник отладки}
\begin{tabular}{|p{50pt}|p{130pt}|p{130pt}|p{70pt}|}
\hline
Дата & Событие & Действие по исправлению & Примечание \\ \hline
21.03.2021 & Ошибка в sum-product2: The value NIL is not of type NUMBER.   & Изменил условие: ((null list1) NIL) в sum-product2 на ((null list1) 0) .  &\\
\hline
\end{tabular}

\section{Замечания автора по существу работы}
Замечаний нет.

\section{Выводы}
В ходе данной работы я познакомился с представлением и основными особенностями списков в common Lisp. Список в common Lisp представляет собой S-выражение вида: атом | список(хвост). Также хотелось бы отметить: в (sum-product1 list1 list2)  реализована хвостовая рекурсия.

\end{document}
