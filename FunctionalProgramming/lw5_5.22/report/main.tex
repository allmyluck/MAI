\documentclass[12pt]{article}

\usepackage{fullpage}
\usepackage{multicol,multirow}
\usepackage{tabularx}
\usepackage{ulem}
\usepackage[utf8]{inputenc}
\usepackage[russian]{babel}
\usepackage{amsmath}
\usepackage{amssymb}

\usepackage{titlesec}

\titleformat{\section}
  {\normalfont\Large\bfseries}{\thesection.}{0.3em}{}

\titleformat{\subsection}
  {\normalfont\large\bfseries}{\thesubsection.}{0.3em}{}

\titlespacing{\section}{0pt}{*2}{*2}
\titlespacing{\subsection}{0pt}{*1}{*1}
\titlespacing{\subsubsection}{0pt}{*0}{*0}
\usepackage{listings}
\lstloadlanguages{Lisp}
\lstset{extendedchars=false,
	breaklines=true,
	breakatwhitespace=true,
	keepspaces = true,
	tabsize=2
}
\begin{document}


\section*{Отчет по лабораторной работе №\,5 
по курсу \guillemotleft  Функциональное программирование\guillemotright}
\begin{flushright}
Студент группы 8О-307Б-18 МАИ \textit{Токарев Никита}, \textnumero 21 по списку \\
\makebox[7cm]{Контакты: {\tt tokarevnikita08@mail.ru} \hfill} \\
\makebox[7cm]{Работа выполнена: 18.04.2021 \hfill} \\
\ \\
Преподаватель: Иванов Дмитрий Анатольевич, доц. каф. 806 \\
\makebox[7cm]{Отчет сдан: \hfill} \\
\makebox[7cm]{Итоговая оценка: \hfill} \\
\makebox[7cm]{Подпись преподавателя: \hfill} \\

\end{flushright}

\section{Тема работы}
Обобщённые функции, методы и классы объектов.

\section{Цель работы}
Изучить  обобщённые функции, методы и классы объектов, а также методы работы с ними в Коммон Лисп.

\section{Задание (вариант № 5.22)}
Определить обычную функцию - предикат line-parallel2-p,
возвращающий T, если два отрезка (экземпляра класса line) параллельны.

Причём концы отрезков могут задаваться как в декартовых (экземплярами cart), так и в полярных координатах (экземплярами polar).


\section{Оборудование студента}
Процессор Intel® Core™ i3-5005U CPU @ 2.00GHz × 4, память: 3,8 Gb, разрядность системы: 64.

\section{Программное обеспечение}
UBUNTU 18.04.5 LTS, компилятор sbcl

\section{Идея, метод, алгоритм}
Идея в том, чтобы сравнивать прямые по значению тангенса угла, так как прямую можно представить в виде y = kx + b(соответсвенно координаты должны быть  декартовые). Также, если прямая расположена вдоль оси x, то при вычислении, k стремиться к бесконечности(то есть происходит деление на ноль), поэтому данный случай также необходимо выделить в отдельный кейс.
\begin{itemize}
\setlength{\itemsep}{-1mm} % уменьшает расстояние между элементами списка
\item{(defmethod cart-x ((p polar))) - В данной функции происходит перевод из полярных координат в декартовые).}
\item{(defmethod line-x-const ((l line))) - x_{1} ?== x_{2}}

\item{(defmethod line-k ((line1 line))) - вычисление k; k = (y_{2}-y_{1}) / (x_{2}-x_{1}).}

\end{itemize}

\section{Сценарий выполнения работы}
\begin{itemize}
\setlength{\itemsep}{-1mm}
\item Анализ возможных реализаций поставленной задачи на common Lisp
\item Изучение синтаксиса и основных функций работы со списками common Lisp
\item Реализация поставленной задачи на common Lisp
\end{itemize}
\section{Распечатка программы и её результаты}

\subsection{Исходный код}
Значения тестовых функций представлено в исходном коде.
\begin{verbatim}
(defclass polar ()
    ((radius :initarg :radius :accessor radius)
    (angle  :initarg :angle  :accessor angle)))

(defclass cart ()
    ((x :initarg :x :reader cart-x)
    (y :initarg :y :reader cart-y)))

(defclass line ()
    ((start :initarg :start :accessor line-start)
    (end   :initarg :end   :accessor line-end)))

(defmethod cart-x ((p polar)) ; angle ~ (-pi,pi]
    (* (radius p) (cos (angle p))))

(defmethod cart-y ((p polar)) ; angle ~ (-pi,pi]
    (* (radius p) (sin (angle p))))

(defmethod line-k ((line1 line))
    (/ (- (cart-y (line-start line1)) (cart-y (line-end line1)))
    (- (cart-x (line-start line1)) (cart-x (line-end line1)))))

(defmethod line-x-const ((l line))
    (almost-equal (cart-x (line-start l)) (cart-x (line-end l))))

(defmethod print-object ((p polar) stream)
    (format stream "[POLAR (radius ~d angle ~d)]"
    (radius p) (angle p)))

(defmethod print-object ((c cart) stream)
    (format stream "[CART (x ~d y ~d)]"
    (cart-x c) (cart-y c)))

(defmethod print-object ((lin line) stream)
    (format stream "LINE (~s ~s)"
    (line-start lin) (line-end lin)))

(defun almost-equal (a b)
    (< (abs(- a b)) 0.001))

(defun line-parallel2-p (line1 line2)
    (if (or (line-x-const line1) (line-x-const line2))
        (if (and (line-x-const line1) (line-x-const line2)) T NIL)
        (if (almost-equal (line-k line1) (line-k line2)) T NIL)))

;(setq l1 (make-instance 'line :start (make-instance 'cart :x 1 :y 1)
:end   (make-instance 'cart :x 0 :y -1)))
;(setq l2 (make-instance 'line :start (make-instance 'cart :x 0 :y 2)
:end   (make-instance 'cart :x -1 :y 0)))
;(setq l2 (make-instance 'line :start (make-instance 'cart :x -1 :y 0)
:end   (make-instance 'cart :x  0 :y 1)))
;(setq l2 (make-instance 'line :start (make-instance 'polar :radius 1 
:angle (/ pi 4)) :end   (make-instance 'cart :x  0 :y 2)))
;(setq l2 (make-instance 'line :start (make-instance 'polar :radius 0 
:angle 0) :end   (make-instance 'cart :x  1 :y 1)))
;(setq l2 (make-instance 'line :start (make-instance 'cart :x 0 :y 2) 
:end   (make-instance 'cart :x 0 :y 0)))
;(setq l2 (make-instance 'line :start (make-instance 'polar :radius 20 
:angle (/ pi 2)) :end   (make-instance 'polar :radius 2 :angle (- (/ pi 2)))))
;(line-parallel2-p l1 l2)
\end{verbatim}
%\lstinputlisting{./lab2.lisp}

\subsection{Результаты работы}
\begin{verbatim}
* (setq l1 (make-instance 'line :start (make-instance 'cart :x 1 :y 1) 
:end   (make-instance 'cart :x 0 :y -1)))

LINE ([CART (x 1 y 1)] [CART (x 0 y -1)])
* (setq l2 (make-instance 'line :start (make-instance 'cart :x 0 :y 2) 
:end   (make-instance 'cart :x -1 :y 0)))

LINE ([CART (x 0 y 2)] [CART (x -1 y 0)])
* (line-parallel2-p l1 l2)

T
* (setq l2 (make-instance 'line :start (make-instance 'cart :x -1 :y 0) 
:end   (make-instance 'cart :x  0 :y 1)))

LINE ([CART (x -1 y 0)] [CART (x 0 y 1)])
* (line-parallel2-p l1 l2)

NIL
* (setq l2 (make-instance 'line :start (make-instance 'polar :radius 1 
:angle (/ pi 4)) :end   (make-instance 'cart :x  0 :y 2)))

LINE ([POLAR (radius 1 angle 0.7853981633974483d0)] [CART (x 0 y 2)])
* (line-parallel2-p l1 l2)

NIL

LINE ([POLAR (radius 0 angle 0)] [CART (x 1 y 1)])
* (line-parallel2-p l1 l2)

NIL
* (setq l2 (make-instance 'line :start (make-instance 'cart :x 0 :y 2) 
:end   (make-instance 'cart :x 0 :y 0)))

LINE ([CART (x 0 y 2)] [CART (x 0 y 0)])
* (setq l1 (make-instance 'line :start (make-instance 'cart :x 3 :y 2) 
:end   (make-instance 'cart :x 3 :y 0)))

LINE ([CART (x 3 y 2)] [CART (x 3 y 0)])
* (line-parallel2-p l1 l2)

T
* (setq l2 (make-instance 'line :start (make-instance 'polar :radius 20 
:angle (/ pi 2)) :end   (make-instance 'polar :radius 2 :angle (- (/ pi 2)))))

LINE ([POLAR (radius 20 angle 1.5707963267948966d0)]
[POLAR (radius 2 angle -1.5707963267948966d0)])
\end{verbatim}
%\lstinputlisting{./log2.lisp}

\section{Дневник отладки}
\begin{tabular}{|p{50pt}|p{130pt}|p{130pt}|p{70pt}|}
\hline
Дата & Событие & Действие по исправлению & Примечание \\ \hline
18.04.2021 & (defmethod line-k ((line1 line))) - ошибка из-за деления на 0 & Добавил поддержку вертикальных линий и дополнил главный предикат условием &\\
\hline
\end{tabular}

\section{Замечания автора по существу работы}
Замечаний нет.

\section{Выводы}
В ходе данной работы мне удалось познакомиться с обобщёнными функциями, методами и классами в common Lisp. Lisp был создан за пару десятилетий до того момента, когда объектно-ориентированное программирование стало популярным. Хотел бы отметить, что  Common Lisp является полноценным объектно-ориентированным языком и в данном языке заложены основные парадигмы/принципы объектно-ориентированного программирования.

\end{document}
